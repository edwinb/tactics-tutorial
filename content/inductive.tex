\section{Inductive Proofs}

\label{sect:induction}

Before embarking on proving \texttt{plus\_commutes} in \Idris{} itself, let
us consider the overall structure of a proof of some property of
natural numbers. Recall that they are defined recursively, as follows:

\begin{code}
data Nat : Type where
     Z : Nat
     S : Nat -> Nat
\end{code}

\noindent
A \emph{total} function over natural numbers must both terminate, and cover
all possible inputs. \Idris{} checks functions for totality by cheking that
all inputs are covered, and that all recursive calls are on \emph{structurally
smaller} values (so recursion will always reach a base case). Recalling
\texttt{plus}:

\begin{code}
plus : Nat -> Nat -> Nat
plus Z     m = m
plus (S k) m = S (plus k m)
\end{code}

\noindent
This is total because it covers all possible inputs (the first argument can
only be \texttt{Z} or \texttt{S k} for some \texttt{k}, and the second 
argument \texttt{m} covers all possible \texttt{Nat}s) and in the recursive
call, \texttt{k} is structurally smaller than \texttt{S k} so the first argument
will always reach the base case \texttt{Z} in any sequence of recursive calls.

In some sense, this resembles a mathematical proof by induction (and this is
no coincidence!). For some property \texttt{P} of a natural number \texttt{x},
we can show that \texttt{P} holds for all \texttt{x} if:

\begin{itemize}
\item \texttt{P} holds for zero (the base case).
\item Assuming that \texttt{P} holds for \texttt{k}, we can show \texttt{P}
also holds for \texttt{S k} (the inductive step).
\end{itemize}

\noindent
In \texttt{plus}, the property we are trying to show is somewhat trivial (for
all natural numbers \texttt{x}, there is a \texttt{Nat} which need not have any
relation to \texttt{x}). However, it still takes the form of a base case
and an inductive step. In the base case, we show that there is a \texttt{Nat}
arising from \texttt{plus n m} when \texttt{n = Z}, and in the inductive
step we show that there is a \texttt{Nat} arising when \texttt{n = S k}
and we know we can get a \texttt{Nat} inductively from \texttt{plus k m}.
%
We could even write a function capturing all such inductive definitions:

\begin{code}
nat_induction : (P : Nat -> Type) ->             -- Property to show
                (P Z) ->                         -- Base case 
                ((k : Nat) -> P k -> P (S k)) -> -- Inductive step
                (x : Nat) ->                     -- Show for all x
                P x
nat_induction P p_Z p_S Z = p_Z
nat_induction P p_Z p_S (S k) = p_S k (nat_induction P p_Z p_S k)
\end{code}

\noindent
Using \texttt{nat\_induction}, we can implement an equivalent inductive
version of \texttt{plus}:

\begin{code}
plus_ind : Nat -> Nat -> Nat
plus_ind n m 
   = nat_induction (\x => Nat)
                   m                      -- Base case, plus_ind Z m
                   (\k, k_rec => S k_rec) -- Inductive step plus_ind (S k) m
                                          -- where k_rec = plus_ind k m
                   n 
\end{code}

\noindent
To prove that \texttt{plus n m = plus m n} for all natural numbers \texttt{n}
and \texttt{m}, we can also use induction. Either we can fix \texttt{m} and
perform induction on \texttt{n}, or vice versa. We can sketch an outline of
a proof; performing induction on 
\texttt{n}, we have:

\begin{itemize}
\item Property \texttt{P} is \texttt{$\backslash$x => plus x m = plus m x}.
\item Show that \texttt{P} holds in the base case and inductive step:
\begin{itemize}
\item Base case: \texttt{P Z}, i.e. \\
\texttt{plus Z m = plus m Z}, which reduces to \\
\texttt{m = plus m Z} due to the definition of \texttt{plus}.
\item Inductive step: Inductively, we know that \texttt{P k} holds for a
specific, fixed \texttt{k}, i.e.\\
\texttt{plus k m = plus m k} (the induction hypothesis). Given this,
show \texttt{P (S k)}, i.e. \\
\texttt{plus (S k) m = plus m (S k)}, which
reduces to \\
\texttt{S (plus k m) = plus m (S k)}. From the induction hypothesis, we can rewrite this to \\
\texttt{S (plus m k) = plus m (S k)}.
\end{itemize}
\end{itemize}

\noindent
To complete the proof we therefore need to show that \texttt{m = plus m Z}
for all natural numbers \texttt{m}, and that \texttt{S (plus m k) = plus m (S k)}
for all natural numbers \texttt{m} and \texttt{k}. Each of these can also
be proved by induction, this time on \texttt{m}.

We are now ready to embark on a proof of commutativity of \texttt{plus}
formally in \Idris{}.

