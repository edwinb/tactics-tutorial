\section{Running example: Addition of Natural Numbers}

Throughout this tutorial, we will be working with the following function,
defined in the \Idris{} prelude, which defines addition on natural numbers:

\begin{code}
plus : Nat -> Nat -> Nat
plus Z     m = m
plus (S k) m = S (plus k m)
\end{code}

\noindent
It is defined by the above equations, meaning that we have for free the
properties that adding \texttt{m} to zero always results in \texttt{m},
and that adding \texttt{m} to any non-zero number \texttt{S k} always results 
in \texttt{S (plus k m)}. We can see this by evaluation at the \Idris{}
REPL (i.e. the prompt, the read-eval-print loop):

\begin{lstlisting}
Idris> \m => plus Z m
\m => m : Nat -> Nat

Idris> \k,m => plus (S k) m
\k => \m => S (plus k m) : Nat -> Nat -> Nat
\end{lstlisting}

\noindent
Note that unlike many other language REPLs, the \Idris{} REPL performs
evaluation on \emph{open} terms, meaning that it can reduce terms which appear
inside lambda bindings, like those above. Therefore, we can introduce unknowns
\texttt{k} and \texttt{m} as lambda bindings and see how \texttt{plus}
reduces.

The \texttt{plus} function has a number of other useful properties, for
example:

\begin{itemize}
\item It is \emph{commutative}, that is for all \texttt{Nat} inputs
\texttt{n} and \texttt{m}, we know that \texttt{plus n m = plus m n}.
\item It is \emph{associative}, that is for all \texttt{Nat} inputs
\texttt{n}, \texttt{m} and \texttt{p}, 
we know that \texttt{plus n (plus m p) = plus (plus m n) p}.
\end{itemize}

\noindent
We can use these properties in an \Idris{} program, but in order to do so we
must \emph{prove} them.

\subsection{Equality Proofs}

\Idris{} has a built-in propositional equality type, conceptually defined as
follows:

\begin{code}
data (=) : a -> b -> Type where
     refl : x = x
\end{code}

\noindent
Note that this must be built-in, rather than defined in the library,
because \texttt{=} is a reserved operator --- you cannot define this directly
in your own code. 

It is \emph{propositional}
equality, where the type states that any two values in different types
\texttt{a} and \texttt{b} may be proposed to be equal.
There is only one way to \emph{prove} equality, however, which is by
reflexivity (\texttt{refl}). 

We have a \emph{type} for propositional equality here, and correspondingly
a \emph{program} inhabiting an instance of this type can be seen as a proof 
of the corresponding proposition\footnote{This is known as the Curry-Howard
corrsepondence~\cite{howard}}.
So, trivially, we can prove that \texttt{4}
equals \texttt{4}:

\begin{code}
four_eq : 4 = 4
four_eq = refl
\end{code}

\noindent
However, trying to prove that \texttt{4 = 5} results in failure:

\begin{code}
four_eq_five : 4 = 5
four_eq_five = refl
\end{code}

\noindent
The type \texttt{4 = 5} is a perfectly valid type, but is uninhabited, so
when trying to type check this definition, \Idris{} gives the following
error:

\begin{code}
When elaborating right hand side of four_eq_five:
Can't unify
        5 = 5
with
        4 = 5
\end{code}

\subsubsection*{Type checking equality proofs}

An important step in type checking \Idris{} programs is \emph{unification},
which attempts to resolve implicit arguments such as the implicit argument
\texttt{x} in \texttt{refl}. As far as our understanding of type checking
proofs is concerned, it suffices to know that unifying two terms involves
reducing both to normal form then trying to find an assignment to implicit
arguments which will make those normal forms equal.

When type checking \texttt{refl}, \Idris{} requires that the type is of
the form \texttt{x = x}, as we see from the type of \texttt{refl}. In the case
of \texttt{four\_eq\_five}, \Idris{} will try to unify the expected type
\texttt{4 = 5} with the type of \texttt{refl},
\texttt{x = x}, notice that a solution requires that \texttt{x} be both
\texttt{4} and \texttt{5}, and therefore fail.

Since type checking involves reduction to normal form, we can write the
following equalities directly:

\begin{code}
twoplustwo_eq_four : 2 + 2 = 4
twoplustwo_eq_four = refl

plus_reduces_Z : (m : Nat) -> plus Z m = m
plus_reduces_Z m = refl

plus_reduces_Sk : (k, m : Nat) -> plus (S k) m = S (plus k m)
plus_reduces_Sk k m = refl
\end{code}

\subsection{Heterogeneous Equality}

Equality in \Idris{} is \emph{heterogeneous}, meaning that 
we can even propose equalities between values in different types:

\begin{code}
idris_not_php : 2 = "2"
\end{code}

\noindent
Obviously, in \Idris{} the type \texttt{2 = "2"} is uninhabited, and one might
wonder why it is useful to be able to propose equalities between values in
different types. However, with dependent types, such equalities can arise
naturally. For example, if two vectors are equal, their lengths must be
equal:

\begin{code}
vect_eq_length : (xs : Vect n a) -> (ys : Vect m a) ->
                 (xs = ys) -> n = m
\end{code}

\noindent
In the above declaration, \texttt{xs} and \texttt{ys} have different types
because their lengths are different,
but we would still like to draw a conclusion about the lengths if they
happen to be equal. We can define \texttt{vect\_eq\_length} as follows:

\begin{code}
vect_eq_length xs xs refl = refl
\end{code}

\noindent
By matching on \texttt{refl} for the third argument, we know that the only
valid value for \texttt{ys} is \texttt{xs}, because they must be equal, and
therefore their types must be equal, so the lengths must be equal.

Alternatively, we can put an underscore for the second \texttt{xs}, since there
is only one value which will type check:

\begin{code}
vect_eq_length xs _ refl = refl
\end{code}

\subsection{Properties of \texttt{plus}}

Using the \texttt{(=)} type, we can now state the properties of \texttt{plus}
given above as \Idris{} type declarations:

\begin{code}
plus_commutes : (n, m : Nat) -> plus n m = plus m n
plus_assoc : (n, m, p : Nat) -> plus n (plus m p) = plus (plus n m) p
\end{code}

\noindent
Both of these properties (and many others)
are proved for natural number addition in the \Idris{}
standard library, using \texttt{(+)} from the \texttt{Num} type class rather
than using \texttt{plus} directly. They have the names \texttt{plusCommutative}
and \texttt{plusAssociatie} respectively.

In the remainder of this tutorial, we will explore several different ways
of proving \texttt{plus\_commutes} (or, to put it another way, writing the
function.) We will also discuss how to use such equality proofs, and see where
the need for them arises in practice.

