\section{Pattern Matching Proofs}

In this section, we will provide a proof of \texttt{plus\_commutes} directly,
by writing a pattern matching definition. We will use interactive editing
features extensively, since it is significantly easier to produce a proof
when the machine can give the types of intermediate values and construct
components of the proof itself. The commands we will use are summarised
in Figure \ref{commands}. Where we refer to commands directly, we will
use the Vim version, but these commands have a direct mapping to Emacs commands.

\newcommand{\mkdef}{\texttt{$\backslash$d}}
\newcommand{\mklem}{\texttt{$\backslash$l}}
\newcommand{\ctype}{\texttt{$\backslash$t}}
\newcommand{\csplit}{\texttt{$\backslash$c}}
\newcommand{\psearch}{\texttt{$\backslash$o}}

\begin{figure}
\begin{tabularx}{\linewidth}{lllX}
\toprule
Command & Vim binding & Emacs binding & Explanation \\
\midrule

Check type & \texttt{$\backslash$t} & \texttt{C-c C-t} &
Show type of identifier or metavariable under the cursor.\\

Proof search & \texttt{$\backslash$o} & \texttt{C-c C-o} &
Attempt to solve metavariable under the cursor by applying simple proof
search.\\

Make new definition & \texttt{$\backslash$d} & \texttt{C-c C-d} &
Add a template definition for the type defined under the cursor.\\

Make lemma & \texttt{$\backslash$l} & \texttt{C-c C-l} &
Add a top level function with a type which solves the metavariable under the
cursor.\\

Split cases & \texttt{$\backslash$c} & \texttt{C-c C-c} &
Create new constructor patterns for each possible case of the variable under
the cursor.\\
\bottomrule
\end{tabularx}
\caption{Interactive Editing Commands}
\label{commands}
\end{figure}

\subsection{Creating a Definition}

To begin, create a file \texttt{pluscomm.idr}
containing the following type declaration:

\begin{code}
total
plus_commutes : (n : Nat) -> (m : Nat) -> n + m = m + n
\end{code}

\noindent
The \texttt{total} annotation means that we aim to write a function which
passes the totality checker; i.e. it will terminate on all possible
well-typed inputs. This is important for proofs, since it provides a
guarantee that the proof is valid in \emph{all} cases, not just those for
which it happens to be well-defined.

To create a template definition for the proof, press \mkdef{}
the equivalent in your editor of choice) on the line with the type
declaration. You should see:

\begin{code}
total
plus_commutes : (n : Nat) -> (m : Nat) -> n + m = m + n
plus_commutes n m = ?plus_commutes_rhs
\end{code}

\noindent
To prove this by induction on \texttt{n}, as we sketched in Section
\ref{sect:induction}, we begin with a case split on \texttt{n} (press
\csplit{}
with the cursor over the \texttt{n} in the definition.) You
should see:

\begin{code}
total
plus_commutes : (n : Nat) -> (m : Nat) -> n + m = m + n
plus_commutes Z m = ?plus_commutes_rhs_1
plus_commutes (S k) m = ?plus_commutes_rhs_2
\end{code}

\noindent
If we inspect the types of the newly created metavariables,
\texttt{plus\_commutes\_rhs\_1}
and
\texttt{plus\_commutes\_rhs\_2}, we see that the type of each reflects that
\texttt{n} has been refined to \texttt{Z} and \texttt{S k} in each respective
case. Pressing \ctype{} over \texttt{plus\_commutes\_rhs\_1} shows:

\begin{code}
  m : Nat
--------------------------------------
plus_commutes_rhs_1 : m = plus m 0
\end{code}

\noindent
Note that \texttt{Z} renders as \texttt{0} because the pretty printer renders
natural numbers as integer literals for readability.
Similarly, for \texttt{plus\_commutes\_rhs\_2}:

\begin{code}
  k : Nat
  m : Nat
--------------------------------------
plus_commutes_rhs_2 : S (plus k m) = plus m (S k)
\end{code}

\noindent
It is a good idea to give these slightly more meaningful names:

\begin{code}
total
plus_commutes : (n : Nat) -> (m : Nat) -> n + m = m + n
plus_commutes Z m = ?plus_commutes_Z
plus_commutes (S k) m = ?plus_commutes_S
\end{code}

\subsection{Base Case}

We can create a separate lemma for the base case interactively, by pressing
\mklem{} with the cursor over \texttt{plus\_commutes\_Z}.
This yields:

\begin{code}
plus_commutes_Z : m = plus m 0

total
plus_commutes : (n : Nat) -> (m : Nat) -> n + m = m + n
plus_commutes Z m = plus_commutes_Z
plus_commutes (S k) m = ?plus_commutes_S
\end{code}

\noindent
That is, the metavariable has been filled with a call to a top level function
\texttt{plus\_commutes\_Z}. The argument \texttt{m} has been made implicit
because it can be inferred from context when it is applied.

Unfortunately, we cannot prove this lemma directly, since \texttt{plus} is
defined by matching on its \emph{first} argument, and here \texttt{plus m 0}
has a specific value for its \emph{second argument} (in fact, the left hand
side of the equality has been reduced from \texttt{plus 0 m}.) Again, we
can prove this by induction, this time on \texttt{m}.

First, create a template definition with \mkdef{}:

\begin{code}
plus_commutes_Z : m = plus m 0
plus_commutes_Z = ?plus_commutes_Z_rhs
\end{code}

\noindent
Since we are going to write this by induction on \texttt{m}, which is
implciit, we will need to bring \texttt{m} into scope manually:

\begin{code}
plus_commutes_Z : m = plus m 0
plus_commutes_Z {m} = ?plus_commutes_Z_rhs
\end{code}

\noindent
Now, case split on \texttt{m} with \csplit{}:

\begin{code}
plus_commutes_Z : m = plus m 0
plus_commutes_Z {m = Z} = ?plus_commutes_Z_rhs_1
plus_commutes_Z {m = (S k)} = ?plus_commutes_Z_rhs_2
\end{code}

\noindent
Checking the type of \texttt{plus\_commutes\_Z\_rhs\_1} shows the following,
which is easily proved by reflection:

\begin{code}
--------------------------------------
plus_commutes_Z_rhs_1 : 0 = 0
\end{code}

\noindent
For such trivial proofs, we can let \Idris{} write the proof automatically
by pressing \psearch{} with the cursor over \texttt{plus\_commutes\_Z\_rhs\_1}.
This yields:

\begin{code}
plus_commutes_Z : m = plus m 0
plus_commutes_Z {m = Z} = refl
plus_commutes_Z {m = (S k)} = ?plus_commutes_Z_rhs_2
\end{code}

\noindent
For \texttt{plus\_commutes\_Z\_rhs\_2}, we are not so lucky:

\begin{code}
  k : Nat
--------------------------------------
plus_commutes_Z_rhs_2 : S k = S (plus k 0)
\end{code}

\noindent
Inductively, we should know that \texttt{k = plus k 0}, and we can get access
to this inductive hypothesis by making a recursive call on k, as follows:

\begin{code}
plus_commutes_Z : m = plus m 0
plus_commutes_Z {m = Z} = refl
plus_commutes_Z {m = (S k)} = let rec = plus_commutes_Z {m=k} in
                                  ?plus_commutes_Z_rhs_2
\end{code}

\noindent
For \texttt{plus\_commutes\_Z\_rhs\_2}, we now see:

\begin{code}
  k : Nat
  rec : k = plus k (fromInteger 0)
--------------------------------------
plus_commutes_Z_rhs_2 : S k = S (plus k 0)
\end{code}

\noindent
Again, the \texttt{fromInteger 0} is merely due to \texttt{Nat} being an
instance of the \texttt{Num} typeclass. So we know that \texttt{k = plus k 0},
but how do we use this to update the goal to \texttt{S k = S k}?

To achieve this, \Idris{} provides a \texttt{replace} function as part of the
prelude:

\begin{code}
*pluscomm> :t replace
replace : (x = y) -> P x -> P y
\end{code}

\noindent
Given a proof that \texttt{x = y}, and a property \texttt{P} which holds for
\texttt{x}, we can get a proof of the same property for \texttt{y}, because
we know \texttt{x} and \texttt{y} must be the same. In practice, this function
can be a little tricky to use because in general the implicit argument
\texttt{P} can be hard to infer by unification, so \Idris{} provides a
high level syntax which calculates the property and applies replace:

\begin{code}
rewrite prf in expr
\end{code}

\noindent
If we have \texttt{prf : x = y}, and the required type for \texttt{expr} is
some property of \texttt{x}, the \texttt{rewrite ... in} syntax will search
for \texttt{x} in the required type of \texttt{expr} and replace it with
\texttt{y}. Concretely, in our example, we can say:

\begin{code}
plus_commutes_Z {m = (S k)} = let rec = plus_commutes_Z {m=k} in
                                  rewrite rec in ?plus_commutes_Z_rhs_2
\end{code}

\noindent
Checking the type of \texttt{plus\_commutes\_Z\_rhs\_2} now gives:

\begin{code}
  k : Nat
  rec : k = plus k (fromInteger 0)
  _rewrite_rule : plus k 0 = k
--------------------------------------
plus_commutes_Z_rhs_2 : S (plus k 0) = S (plus k 0)
\end{code}

\noindent
Using the rewrite rule \texttt{rec} (which we can see in the context here as
\texttt{\_rewrite\_rule}\footnote{Note that the left and right hand sides of the
equality have been swapped, because \texttt{replace} takes a proof of
\texttt{x=y} and the property for \texttt{x}, not \texttt{y}.}, the goal type
has been updated with \texttt{k} replaced by \texttt{plus k 0}.

Alternatively, we could have applied the rewrite in the other direction using
the \texttt{sym} function:

\begin{code}
*pluscomm> :t sym
sym : (l = r) -> r = l
\end{code}

\begin{code}
plus_commutes_Z {m = (S k)} = let rec = plus_commutes_Z {m=k} in
                                  rewrite sym rec in ?plus_commutes_Z_rhs_2
\end{code}

\noindent
In this case, inspecting the type of the hole gives:

\begin{code}
  k : Nat
  rec : k = plus k (fromInteger 0)
  _rewrite_rule : k = plus k 0
--------------------------------------
plus_commutes_Z_rhs_2 : S k = S k
\end{code}

\noindent
Either way, we can use proof search (\psearch{}) to complete the proof,
giving:

\begin{code}
plus_commutes_Z : m = plus m 0
plus_commutes_Z {m = Z} = refl
plus_commutes_Z {m = (S k)} = let rec = plus_commutes_Z {m=k} in
                                  rewrite rec in refl
\end{code}

\noindent
The base case is now complete.

\subsection{Inductive Step}

Our main theorem, \texttt{plus\_commutes} should currently be in the following
state:

\begin{code}
total
plus_commutes : (n : Nat) -> (m : Nat) -> n + m = m + n
plus_commutes Z m = plus_commutes_Z
plus_commutes (S k) m = ?plus_commutes_S
\end{code}

\noindent
Looking again at the type of \texttt{plus\_commutes\_S}, we have:

\begin{code}
  k : Nat
  m : Nat
--------------------------------------
plus_commutes_S : S (plus k m) = plus m (S k)
\end{code}

\noindent
Conveniently, by induction we can immediately tell that \texttt{plus k m = plus m k},
so let us rewrite directly by making a recursive call to \texttt{plus\_commutes}.
We add this directly, by hand, as follows:

\begin{code}
total
plus_commutes : (n : Nat) -> (m : Nat) -> n + m = m + n
plus_commutes Z m = plus_commutes_Z
plus_commutes (S k) m = rewrite plus_commutes k m in ?plus_commutes_S
\end{code}

Checking the type of \texttt{plus\_commutes\_S} now gives:

\begin{code}
  k : Nat
  m : Nat
  _rewrite_rule : plus m k = plus k m
--------------------------------------
plus_commutes_S : S (plus m k) = plus m (S k)
\end{code}

\noindent
The good news is that \texttt{m} and \texttt{k} now appear in the correct order.
However, we still have to show that the successor symbol \texttt{S} can be moved
to the front in the right hand side of this equality. This remaining lemma
takes a similar form to the \texttt{plus\_commutes\_Z}; we begin by making
a new top level lemma with \mklem{}. This gives:

\begin{code}
plus_commutes_S : (k : Nat) -> (m : Nat) -> S (plus m k) = plus m (S k)
\end{code}

\noindent
Unlike the previous case, \texttt{k} and \texttt{m} are not made implicit
because we cannot in general infer arguments to a function from its result.
Again, we make a template definition with \mkdef{}:

\begin{code}
plus_commutes_S : (k : Nat) -> (m : Nat) -> S (plus m k) = plus m (S k)
plus_commutes_S k m = ?plus_commutes_S_rhs
\end{code}

\noindent
Again, this is defined by induction over \texttt{m}, since \texttt{plus}
is defined by matching on its first argument. The complete definition is:

\begin{code}
plus_commutes_S : (k : Nat) -> (m : Nat) -> S (plus m k) = plus m (S k)
plus_commutes_S k Z = refl
plus_commutes_S k (S j) = rewrite plus_commutes_S k j in refl
\end{code}

\noindent
All metavariables have now been solved, and \texttt{plus\_commutes} has
a \texttt{total} annotation, so we have completed the proof of commutativity
of addition on natural numbers.
